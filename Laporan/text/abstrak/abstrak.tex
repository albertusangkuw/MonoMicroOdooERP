\chapter*{Abstrak}

\begin{longtable}{@{}p{2.5cm} l p{10.3cm}}
	Nama 			& : & Albertus Septian Angkuw \\
	Program Studi	& : & Informatika \\
	Judul			& : & Penerapan  \textit{Microservice} dengan \textit{Hierarchical Clustering} untuk Dekomposisi dari Monolitik pada \textit{Enterprise Resource Planning} \\
\end{longtable}
 
 Aplikasi \textit{Enterprise Resource Planning} (ERP) dapat dibangun dengan arsitektur Monolitik, \textit{Service Oriented Architecture}, dan \textit{Microservice}. Arsitektur monolitik merupakan arsitektur sederhana namun monolitik tidak mudah dilakukan \textit{scaling} dan sulit dikembangkan secara berkelanjutan sedangkan \textit{Microservice} merupakan arsitektur modern yang cocok pada aplikasi perusahaan yang telah tumbuh dengan skala secara vertikal maupun horizontal. Manfaat dari \textit{microservice} membuat   perusahaan melakukan migrasi aplikasi berarsitektur monolitik menjadi arsitektur \textit{microservice}. Namun proses ini terbukti sulit dan mahal, salah satu tantangan adalah bagaimana mengidentifikasi komponen dari aplikasi monolitik. Identifikasi dapat dilakukan secara semi-otomatis yang menggunakan algoritma \textit{clustering}. Algoritma \textit{clustering} yang digunakan yaitu \textit{Hierarchical} \textit{Clustering} dimana terdapat \textit{linkage} seperti \textit{single linkage}, \textit{complete linkage}, dan \textit{average linkage}.
Penelitian ini menggunakan \textit{Odoo} sebagai aplikasi ERP yang dilakukan dekomposisi dari monolitik menjadi \textit{Microservice} dengan pendekatan menganalisis graph dari kode program kemudian memasukan graph ke Hierarchical \textit{Clustering}. Hasil dari pengelompokan diuji dengan melihat cohesion dan coupling untuk setiap \textit{linkage} kemudian dilakukan pemilihan bagian yang di implementasikan. Berdasarkan pengujian ditemukan \textit{linkage} yang cocok untuk membuat kelompok \textit{service} yang memiliki coupling dan cohesion yang baik adalah Average linkage. Pemilihan jumlah \textit{service} yang ideal pada \textit{Average linkage} dimulai dari 175-245 \textit{service}. Dari struktur tabel yang terbentuk ketika implementasi menunjukan \textit{Hierarchical} \textit{Clustering} dapat memisahkan module yang tidak terhubung dan \textit{service} yang memiliki hubungan kuat, seperti pada kasus di partisi ke-10 dengan Module Product dan Module Point of Sale dan pada kasus di partisi ke-17 dengan Module Calendar.

\noindent Kata kunci: Dekomposisi, \textit{Hierarchical Clustering}, \textit{Microservice}, Monolitik, \textit{Enterprise Resource Planning}, \textit{Odoo}