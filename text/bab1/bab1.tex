\chapter{PENDAHULUAN}
\section{Latar Belakang}
Sistem Enterprise Resource Planning(ERP) merupakan perangkat lunak yang digunakan pada perusahaan untuk menjalankan bisnisnya dimana perusahaan dapat mengotomisasi dan mengintegrasi sebagian besar proses bisnisnya dengan ini perusahaan bisa menghasilkan serta mengakses informasi secara langsung. Agar perusahaan tetap kompetitif, perusahaan harus meningkatkan proses bisnisnya seperti dalam hal berbagi informasi dengan produsen, distributor dan pelanggan.\cite{1}\\

Pada penerapannya sistem ERP dilakukan beberapa penyesuaian yang selalu diperlukan untuk memenuhi persyaratan fungsionalitas dan proses dari perusahaan. Selain itu diharapkan juga sistem memiliki skalabilitas terhadapat operasi bisnis, skalabilitas diperlukan dalam jangka panjang, misalkan ketika perusahaan tumbuh dari waktu ke waktu, serta dalam waktu singkat,  misalnya pada saat volume transaksi tinggi seperti belanja Natal.\cite{2}\\

Dalam membangun sistem ERP ini dibutuhkan arsitektur, terdapat beberapa arsitektur yang digunakan dalam membangun sistem ERP yaitu {\it two-tier} , {\it three-tier/n-tier}, dan {\it service oriented architecture(SOA)} dimana arsitektur ini disebarkan secara monolitik. Saat ini arsitektur yang umum digunakan yaitu SOA karena dapat membantu perancangan ERP menjadi lebih terukur, andal dan fleksibel dengan memecah fungsionalitas menjadi bagian kecil yang dinamakan {\it service}.\cite{1}\\

Tetapi service ini memiliki Enterprise Service Bus(ESB) yang merupakan bagian integrasi dan menghubungkan seluruh {\it backend}. Meskipun ESB memiliki keuntungan dalam melakukan pemeriksaan status, melakukan perutean untuk service {\it backend}, meskipun demikian ditemukan bahwa ESB menjadi rumit dan menyebabkan terjadinya {\it bottleneck}. Arstiktur Microservice(MSA) dapat menangani kekurangan ini dengan terisolasi, independen dan mudah didistribusikan sehingga memudahkan skalabilitas. Keuntungan terbesarnya yaitu aplikasi bisa dibangun dengan berbagai pilihan teknologi dan memungkinkannya untuk digunakan secara independen satu sama lain.
Ini sangat menyederhanakan siklus pengembangan, pengujian, pembuatan, dan penerapan aplikasi karena perubahan terbatas pada satu service daripada seluruh aplikasi.\cite{3}\\

Hal ini dibuktikan juga dengan penelitian sebelumnya yang melakukan uji perfoma berupa uji beban dari setiap arsitektur. Dimana MSA memiliki troughput yang lebih tinggi pada 1500 pengguna dengan nilai rata-rata 1,1 dibandingkan dengan arsitektur SOA sebesar 0,7 dan monolith sebesar 0,6. Selain itu pada response time MSA lebih cepat 5 detik yaitu sebesar 33 detik dibandingkan dengan monolith sebesar 38 detik dan SOA sebesar 43 detik. Pada pengukuran jumlah kode response 200(Berhasil), MSA memiliki jumlah response tertinggi di kode berhasil dengan memiliki jumlah response terkecil di kode 302(Pengalihan), 304(Cache) ,408(Waktu Habis), 500(Kesalahan Internal Server) dan tidak memiliki jumlah response di kode 404(Tidak ditemukan). Dimana pada aspek pemeliharaan aplikasi, MSA lebih unggul daripada SOA dan monolit unggul daripada SOA.\cite{4}\\

Naman manfaat ini hanya dapat dimanfaatkan jika backend service diperiksa dengan cermat dan didekomposisi dengan cara yang paling optimal dengan mempertimbangkan gambaran besar dari seluruh cakupan aplikasi. Jika tidak, desain ini mungkin terbukti kontraproduktif dan menyebabkan latensi, kompleksitas, dan inefisiensi Hal ini diperlukan untuk memisahkan sistem menjadi bagian-bagian yang sesuai secara fungsional dan memperoleh service kohesif tinggi dan service yang digabungkan secara longgar diharapkan sebagai hasil dari dekomposisi.\cite{3}\\

Dalam melakukan dekomposisi bisa dilakukan dengan konsep Domain Driven Design(DDD), Functional, Dataflow, dan Dependency Capturing dengan Clustering-Based Microservice. Pada hasil evaluasi DDD menunjukkan sistem berhasi didekomposisi ke microservice. Dengan pendekatan Functional hasil evaluasi menunjukkan bahwa identifikasi microservice dapat dilakukan lebih cepat. Di pendekatan dataflow ini menunjukkan dekomposisi bisa ditentukan dari pertimbangan coupling dan cohesive. Indentikasi microservice dengan menganalisis ketergantungan proses bisnis dari control, dengan data dan control, data, dan semantic models. Kemudian untuk metode Clustering untuk mengindentifikasi microservice, metode clustering yang digunakan yaitu Hierarchical Clustering. Hasil dari validasi pendekatan ini menunjukkan bahwa pendekatan ini mencapai hasil yang lebih baik daripada pendekatan yang ada dalam hal identifikasi microservice.\cite{6}\\

Pada penelitian ini akan melakukan dekomposisi sistem ERP yang disebarkan secara monolitik menjadi arsitektur microservice dengan pendekan menggunakan semantic models Kemudian dilakukan pengelompokan melalui Hierarchical Clustering. Hasil dari pengelompokan akan di implementasikan
dan dilakukan uji beban sehingga mengetahui nilai latensi, jumlah troughput, dan total response. Dengan ini diharapkan bisa menyelesaikan permasalahan yang terjadi di sistem ERP seperti kustomisasi dan skalabilitas.\\

\section{Rumusan Masalah}
Berikut adalah rumusan masalah yang dibuat berdasarkan latar belakang diatas.
\begin{enumerate}[nolistsep,leftmargin=0.5cm]
  \item Bagaimana melakukan dekomposisi dari arsitektur monolitik ke arsitektur microservice?
  \item Bagaimana performa aplikasi antara arsitektur monolitik dan arsitektur microservice dalam kondisi beban yang tinggi?\\
\end{enumerate}

\section{Tujuan Penelitian}
Berdasarkan rumusan masalah di atas, maka tujuan tujuan penelitian ini adalah.
\begin{enumerate}[nolistsep,leftmargin=0.5cm]
  \item Mengetahui dampak perfoma ERP yang menggunakan arsitektur Microservice.
  \item Menerapkan dekomposisi aplikasi monolitik ke microservice dengan pendekatan Hierarchical Clustering\\
\end{enumerate}

\section{Batasan Masalah}
Agar penelitian ini menjadi lebih terarah, maka penulis membatasi masalah yang akan dibahas sebagai berikut.
\begin{enumerate}[nolistsep,leftmargin=0.5cm]
  \item Penyebaran aplikasi dilakukan dengan framewrok docker.
  \item Aplikasi yang digunakan adalah aplikasi yang sudah dibangun sebelumnya dengan arsitektur monolitik.
  \item Perubahan arsitektur tidak menambah atau mengurangi fungsionalitas dari aplikasi.\\
\end{enumerate}

\section{Konstribusi Penelitian}
Kontribusi yang diberikan pada penelitian ini adalah sebagai berikut.
\begin{enumerate}[nolistsep,leftmargin=0.5cm]
  \item Memberikan langkah dalam melakukan dekomposisi aplikasi monolitik ke microservice dengan Hierarchical Clustering  
  \item Melihat pengaruh dari perfoma aplikasi yang sudah dilakukan dekomposisi.
  \item Melakukan pengukuran dengan beban test pada aplikasi.\\
\end{enumerate}

\section{Metodologi Penelitian}
Tahapan-tahapan yang akan dilakukan dalam pelaksanaan penelitian ini adalah sebagai berikut.
\begin{enumerate}[nolistsep,leftmargin=0.5cm]
  \item Penelitian Pustaka
  \item Analisis
  \item Perancangan
  \item Implementasi
  \item Pengujian \\
\end{enumerate}

\section{Sistematika Pembahasan}
Bab 1: PENDAHULUAN
Suspendisse ac porta diam, ut viverra ante. Aliquam mattis tincidunt diam in molestie. Sed auctor fermentum turpis, sed varius ante.\\

Bab 2: LANDASAN TEORI
Suspendisse ac porta diam, ut viverra ante. Aliquam mattis tincidunt diam in molestie. Sed auctor fermentum turpis, sed varius ante.\\

Bab 3: ANALISIS DAN PERANCANGAN
Suspendisse ac porta diam, ut viverra ante. Aliquam mattis tincidunt diam in molestie. Sed auctor fermentum turpis, sed varius ante.\\

Bab 4: IMPLEMENTASI DAN PENGUJIAN
Suspendisse ac porta diam, ut viverra ante. Aliquam mattis tincidunt diam in molestie. Sed auctor fermentum turpis, sed varius ante.\\

Bab 5: KESIMPULAN DAN SARAN
Suspendisse ac porta diam, ut viverra ante. Aliquam mattis tincidunt diam in molestie. Sed auctor fermentum turpis, sed varius ante.\\